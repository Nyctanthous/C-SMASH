%% Requires compilation with XeLaTeX or LuaLaTeX
\documentclass[10pt,xcolor={table,dvipsnames},t]{beamer}
\usetheme{UCBerkeley}

\title[SMACC]{SMACC}
\subtitle{SDSS MOC4 Asteroid Color Classification}
\author{Ben Montgomery, Daniel Shepard, Emily Wall}
\institute{University of Southern Maine}
\date{\today}

\begin{document}

\begin{frame}
  \titlepage
  % That's right, this is the SMACC talk.
\end{frame}

\section{Introduction}

\begin{frame}{Introduction}
\begin{itemize}
    \item The Sloan Digital Sky Survey is a multi-spectral survey of celestial objects, most notably of asteroids.
    \item About a half-million observations of asteroids have been made over the course of 18 years.
    \item Classifying the compositions of these asteroids has been something of a hot topic.
\end{itemize}
\end{frame}

\begin{frame}{Notable Previous Efforts}
\begin{itemize}
    \item \textit{Solar System Objects in the {SDSS} Commissioning Data} (2000): Used Autoclass, a NASA-developed unsupervised Bayesian classification algorithm. Concluded that there were \textbf{two classes}: $C-$ and $S-$ classes \cite{Autoclass}.
    \item \textit{Searching for V-Type and Q-Type Main-Belt Asteroids Based On SDSS Colors} (2007): Used extensive computational and numerical analysis to conclude there were \textbf{four classes}: $C-$, $S-$ $V-$, and $Q-$ classes \cite{binzal_colors}.
    \item \textit{SDSS-based taxonomic classification and orbital distribution of main belt asteroids} (2009): Used templates of what different classes ``should look like'', applied supervised learning, and concluded there were \textbf{sixteen classes} \cite{Carvano}.
    % If you notice a huge jump, you aren't alone. Carvano's results were far better than any other study to date that just uses the SDSS.
    % Some of the professors here doubt the validity of Carvano's study, and think that he overfit his data in order to maximize the number of unique classes.
\end{itemize}
\end{frame}


\begin{frame}{What We're Doing}

We plan to use at least a K-Means++ algorithm and Carvano's data correction methods to see if we can reasonably replicate his results using only unsupervised learning.
\begin{itemize}
    \item If we keep the growth trend from before, we might find out that there are actually 256 classes.
    \item More realistically, we expect to find that there are less than sixteen classes.
    \item It is the belief of some professors here that it is only possible to identify five or six, given the kind of color information in the database.
\end{itemize}
\end{frame}

\begin{frame}
\frametitle{References}

\bibliographystyle{IEEEtran}
\bibliography{references}

\end{frame}


\end{document}
